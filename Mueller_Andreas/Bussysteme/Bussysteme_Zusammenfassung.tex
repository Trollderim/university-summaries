\documentclass{article}

\usepackage{amsmath}
\usepackage{acronym}
\usepackage[utf8]{inputenc}

\newcommand{\myparagraph}[1]{\paragraph{#1}\mbox{}\\}
\newcommand{\follows}{$\rightarrow$ }
\newcommand*\xor{\mathbin{\oplus}}
\newcommand{\tab}[1]{\hspace{.2\textwidth}\rlap{#1}}

\begin{document}
	
	\begin{acronym}
		\acro{FDMA}{Frequency Domayn Multiple Access}
		\acro{RTT}{Roundtriptime}
	\end{acronym}
	
	\title{Bussysteme - Eine Zusammenfassung}
	\author{Phillipp Mevenkamp}
	
	\maketitle
	
	\section{Grundlagen}
	\subsection{Hierarchie}
	- Ethercat / Industrial Ethernet \\	
	- Profibus \\	
	- CAN / LIN \\
	- AS/I \\
	- RS 232 / EIA 485
	
	\subsection{Aufteilung Medium}
	- Datenrate wird in kleiner Einheiten aufgeteilt \\
	- Jede Station enthält dabei eine Einheit zur exklusiven Benutzung
	
	\myparagraph{Wahlfreier Zugriff (Random Access)}
	- Datenrate nicht unterteilt \\
	- Kollisionen können vorkommen \follows Behandlung notwendig
	
	\myparagraph{Abwechselnd}
	- Zugriffe werden koordiniert \follows Kollisionen werden vermieden
	
	\myparagraph{TDMA - Time Division Multiple Access}
	- Zugriff auf Medium in Runden \\
	- Jede Station hat einen festen Zeitschlitz \\
	- nicht verwendete gehen dabei verloren
	
	\myparagraph{\ac{FDMA}}
	- Spektrum des Mediums wird auf Frequenzen aufgeteilt \\
	- Jede Station hat fixe Frequenz \\
	- Station sendet nicht \follows Bereich nicht verwendet
	
	\subsection{Buszugriffsverfahren}
	
	\myparagraph{Master/Slave (AS-I, Interbus, Profibus)}
	- Master ruft Slaves in Reihenfolge auf und fordert Nachricht an \\
	- Abarbeitung in Zyklen \\
	- Jeder Slave hat eigene Adresse \\
	- Slaves stellen Linie dar \\
	- Knotenpunkte galvanisch entkoppelt
	
	\myparagraph{Token Passing}
	- verwendet bei mehreren Mastern \\
	- Zugriffsberechtigung über strenges Token-Passing \\
	- Token wird von Master zu Master gereicht
	
	\myparagraph{CSMA/CD Carrier Sense Multiple Access / Collision Detection}
	- Jeder Teilnehmer gleichberechtigt \\
	- Teilnehmer senden adressierte Telegramme an andere Teilnehmer \\
	- Kollision \follows beide ziehen sich zurück und versuchen Zugriff nach Zufallszeit nochmal \\
	- nicht echtzeitfähig
	- bei wenig Auslastung \follows gut \\
	- bei viel Auslastung \follows viele Kollisionen \\
	- Sendevorgang während \ac{RTT} noch bestehen \follows \ac{RTT} = $$2 * s_{max} = v * t_{frame} = c * VKF * S_{Frame} * (1 / v_{bit})$$
	
	\myparagraph{CSMA/CA Carrier Sense Multiple Access / Collision Avoidance}
	- Jeder Knoten gleichberechtigt \\
	- Senden an alle \\
	- Botschaft mit Identifier versehen \follows Jeder Empfänger behandelt Nachricht je nach ID \\
	- ID legt Priorität für Nachricht bei Kollision fest \follows bedingt echtzeitfähig \\
	- Beispiel: WLAN
	\myparagraph{CSMA/CA Carrier Sense Multiple Access / Collision Resolution}
	- Kommt bei Feldbussen wie CAN zum Einsatz \\
	- Ähnlich CSMA/CA \\
	- Kollisionsvermeidung durch Bitarbitrierung \\
	- In Arbitrierungsphase setzt sich Teilnehmer mit höchster Nachrichtenpriorität durch \follows Senden annähernd verzögerungsfrei
	
	\myparagraph{Schieberegister}
	- Einzelnachrichten für jeden Slave in topologischer Reihenfolge\\
	- Telegramm Abbild der Slave-Anordnung \\
	- Slave nimmt seine (die vorderste) Nachricht und hängt Antwort hinten an \\
	- Slave durch Platz definiert \follows keine Adresse notwendig
	
	\myparagraph{FTDMA - Flexible Time Division Multiple Access}
	- Steuerung basiert auf "Slot-Zählern" in Netzknoten \\
	- Zu Beginn der Übertragung Synchronisation \\
	- Zählerstand = Identifier \\
	- Falls Zählerstand = ID einer Nachricht \follows Nachricht erhält Buszugriff \\
	- Senden \follows Stop der Zähler \\
	
	\myparagraph{Systematisierung der Zugriffsverfahren}
	\begin{tabbing}
	- \= Deterministisch \\
	\> - Zentral gesteuert (Master-Slave) \\
	\> - Dezentral gesteuert (Token, TDMA Verfahren) \\
	- Zufällig \\
	\> - Nicht Kollisionsfrei \\
	\> - Kollisionsfrei (CSMA/CD, CA, CR)
	\end{tabbing}
	
	
	\subsection{Telegramme}
	= binär codierte Nachrichteneinheiten auf Busleitung \\
	- Form liefert Aussage über Leistungsfähigkeit \\
	- Effizienz = Datenbits / Gesamtzahl Bits \\
	- alle Telegramme haben: Start-/Enderahmen, Prüfzeichen (Parity, CRC) und Steuerzeichen
	
	\subsection{Protokoll}
	= vollständiges und eindeutiges Regelwerk für die Kommunikation \\
	Syntax = Zeichen- oder Wortfolgen für Kommunikation \\
	Semantik = richtige Verwendung syntaktisch korrekter Konstrukte \\
	Latenzzeit = Dauer von Versenden bis Empfang einer Nachricht \\
	Übertragungsrate (Brutto/Netto) \\
	\textbf{teilnehmerorientiert} Nachrichten enthalten eine eindeutige Identifikation der Teilnehmer \\
	\textbf{nachrichtenorientiert} Nachrichten sind durch eindeutige Kennung identifiziert - Teilnehmer entscheiden über Verwendung
	
	\subsection{Kommunikationsformen}
	Punkt-zu-Punkt = Ein Sender, ein Empfänger \\
	Anycast = Ein Sender, ein (beliebiger Empfänger) \\
	Broadcast = Ein Sender, potenziell viele Empfänger \\
	Multicast = Ein Sender, viele Empfänger
	
	\myparagraph{Client - Server}
	- Anforderungen (Request)
	- Anzeige (Indication)
	- Antwort (Response)
	- Bestätigung (Confirmation)
	
	\myparagraph{Producer - Consumer}
	- meist Verwendung einer Queue
	- Broadcast
	- Multicast
	
	\myparagraph{Publisher - Subscriber}
	- Multicast an Subscriber
	- Subscriber bekommt Informationen nur, wenn er "abonniert" hat
	
	\subsection{Bitcodierung}
	\myparagraph{Manchester}
	- Triggern auf Flanken
	
	\myparagraph{NRZ - Non-Return-to-Zero}
	- Gleichstrom \follows Maßnahme zur Synchronisation notwendig \\
	\follows Bitstuffing = nach bestimmter Anzahl gleicher Bits, wird Stuffbit eingefügt (beispielsweise 5).
	
	Sonderformen:
	NRZ-L - normale NRZ Codierung \\
	NRZ-I - Invertiere bei Bit den Ursprungspegel \\
	NRZ(I)-M - Bitwechsel bei "1", bei "0" kein Bitwechsel
	$$ p_{k} = d_{k} \xor p_{k-1} $$
	NRZ(I)-S - Bitwechsel bei "0", bei "1" kein Bitwechsel (Verwendung bei USB)
	$$ p_{k} = \neg d_{k} \xor p_{k-1} $$
	
	\subsection{Fehlererkennung/Korrektur - Datensicherung}
	- Werden eingesetzt, um das korrekte Empfangen und Senden einer Nachricht zu prüfen

	\myparagraph{Paritätsprüfung}
	- Bits werden addiert \\
	- Quersumme ist entweder gerade oder ungerade \\
	- Wird durch Bit am Ende der Nachricht signalisiert \\
	\follows Fehlererkennung bei 1-Bit Fehlern
	
	\myparagraph{CRC - Cyclic Redundancy Check}
	- erzeugen eines Generatorpolynoms (z.B. 110101 = $G = x^5 + x^4 + x^2 + x^0$) \\
	- Datenrahmen erweitern um $n = r - 1$ | $r =$ Länge $G$ \\
	- Division des Datenrahmens durch $G$ \\
	- Senden Datenrahmen plus Rest \\
	- Bei Empfangen: Division durch Generatorpolynom \\
	- Richtig empfangen, wenn hier Rest = 0
	
	\myparagraph{Hammingcode}
	
	\subsection{Fehlerkontrolle}
	\myparagraph{passive}
	- bei korrekter Nachricht Quitting an Sender \\
	- Abwarten einer bestimmten Zeit, bis neue Nachricht \\
	- bei keiner Quittung \follows nochmaliges Senden \\
	\follows teilnehmerorientierte Protokolle
	
	\myparagraph{aktive}
	- bei Fehler sendet Empfänger Fehlersignal \\
	\follows nachrichtenorientierte Protokolle
	
	\subsection{Bedeutung der einzelnen Schichten nach OSI}
	\myparagraph{Application Layer}
	- Datenübertragung zwischen "benachbarten" Teilnehmern \\
	- Steuerung Datenfluss \\
	- Realisierung Zugriffsverfahren \\
	- sichert Teilstrecke durch Fehlererkennung und -korrektur
	
	\myparagraph{Data-Link Layer}
	- allgemein verwendbare Grundfunktionen (z.B. Auf- und Abbau von Verbindungen) \\
	- organisatorische Funktionen notwendig \follows Management, bzw. Netzwerkmanagement
	
	\section{Automotive}
	\subsection{Anforderungen}
	- Einteilung nach SAE-Klassen \\ \\
	\begin{tabular}{l | l | l}
		SAE-Klasse & Merkmale & Typisches System \\ \hline
		A & \parbox[t]{7cm}{Vernetzung von Aktoren und Sensoren \\
			Geringe Datenrate (10kbit/s) \\
			Geringe Ausprägung Fehlererkennung und -korrektur} & LIN-Bus \\ \hline
		B & \parbox[t]{7cm}{Vernetzung Steuergeräte \\
			Mittlere Datenraten (125kbit/s) \\
			Komplexe Mechanismen Fehlererkennung und -korrektur} & CAN-Bus \\ \hline
		C & \parbox[t]{7cm}{Vernetzung Steuergeräte mit "einfachen" Echtzeitanforderungen \\
			Hohe Datenraten} & CAN-Bus \\ \hline
		D & \parbox[t]{7cm}{Multimedia-Anwendungen \\
			Hohe Datenraten} & MOST-Bus\\
	\end{tabular}
	
	\section{CAN - Controller Area Network}
	\subsection{Überblick}
	Codierung: NRZ-I \\
	Protokoll: nachrichtenorientiert \\
	Buszugriff: Multi-Master mit Broad- und Multicast, CSMA/CR
	
	\subsection{Arbitrierung}
	- 11 Bit Identifier \\
	- Low-Pegel dominant \\
	\follows Nachricht mit stärkerem Null-Pegel setzt sich durch
	
\end{document}